\section{Introduction}

A community quickly adopts a web application if it complies with user requirements.
To do so, the monolithic programming approach facilitates quick enhancements in features.
Java is a very popular choice using this approach.
To cope with the community growth, the resources shall grow proportionally.
The threading approach to scale an application on growing resources is error-prone.
Eventually this growth requires to replace the initial approach for models providing incremental scalability.
These models generally distribute application parts on a cluster of commodity machines\cite{Fox1997}.
However these models are specific and require the development team to be trained and to start over the initial code base.
These modifications imply to spend development resources in background without adding visible value for the users.

\textit{Node.js} provides a more efficient approach for network applications than the threading approach.
Its monolithic approach facilitates quick enhancements in features, but lacks incremental scalability.
It is based on an event-loop consuming messages, similarly the models described above.
Because of this similarity, we think it is possible to identify autonomous parts communicating through data streams.

To lift the risks described above, we maintain the \textit{Node.js} monolithic approach and propose a tool to compile it into a high-level language compatible with a more scalable model.
It extracts autonomous parts by searching for rupture points marking them out.
% This tool may enable incremental scalability without unsafe technological shifts.

We present rupture points in section \ref{section:rupture}, the high-level language in section \ref{section:language}, and a compilation example in \ref{section:example}.
Finally, we cite related works in section \ref{section:related} and conclude this presentation.