\section{Rupture points} \label{section:rupture}

A \textbf{rupture point} is a call of a loosely coupled function.
It indicates an interface between two application parts along a data stream.
In \textit{Node.js}, I/O operations are asynchronous functions.
That is a function call that resumes immediately, with a function to process later the result of the operation : the callback.
The callback is loosely coupled with the initial caller.
An asynchronous call indicates a rupture point.
To detect rupture points, the compiler uses a predefined list of asynchronous functions.
For example \texttt{app.get} and \texttt{fs.readFile} in listing \ref{lst:source}, lines 5 and 6.

\includecode{js,
  caption={A simple application presenting two asynchronous functions : \texttt{app.get} and \texttt{fs.readFile}},
  label={lst:source}
}
{snippets/source.js}

In \textit{Node.js}, there is a single thread of execution to avoid concurrent memory access.
The compiler needs to control the memory to assure independence between application parts.
If an application part reads a variable from another part, it sends the variable downstream.
If an application part modifies a variable from another part, it persists the variable.
If several application parts modify the same variable, the compiler merges these parts.

\section{High-level language} \label{section:language}

\vspace{-1\baselineskip}
\begin{bnf*}
  \bnfprod{program}    {\bnfpn{flx} \bnfor \bnfpn{flx} \bnfsp \bnftd{eol} \bnfsp \bnfpn{program}}\\
  \bnfprod{flx}    {\bnfts{\texttt{flx}} \bnfsp \bnfpn{id} \bnfsp \bnfpn{ctx} \bnfsp \bnftd{eol} \bnfsp \bnfpn{streams} \bnfsp \bnftd{eol} \bnfsp \bnfpn{fn}}\\
  \bnfprod{streams}    {\bnfts{\texttt{null}} \bnfor \bnfpn{stream} \bnfor \bnfpn{stream} \bnfsp \bnftd{eol} \bnfsp \bnfpn{streams}}\\
  \bnfprod{stream}     {\bnfpn{op} \bnfsp \bnfpn{dest} \bnfsp [\bnfpn{msg}]}\\
  \bnfprod{dest}       {\bnfpn{list}}\\
  \bnfprod{ctx}        {\bnfts{\texttt{\{}} \bnfpn{list} \bnfts{\texttt{\}}}}\\
  \bnfprod{msg}        {\bnfts{\texttt{[}} \bnfpn{list} \bnfts{\texttt{]}}}\\
  \bnfprod{list}       {\bnfpn{id} \bnfor \bnfpn{id} \bnfsp \bnfts{,} \bnfsp \bnfpn{list}}\\
  \bnfprod{op}         {\bnfts{\texttt{>}\texttt{>}} \bnfor \bnfts{\texttt{-}\texttt{>}}}\\
  \bnfprod{id}         {\bnftd{Javascript identifier}}\\
  \bnfprod{fn}         {\bnftd{Javascript and stream syntax}}\\
\end{bnf*}
\vspace{-1.5\baselineskip}~\\
The compiler encapsulates an application part in a \textit{fluxion} $\bnfpn{flx}$.
It is an autonomous entity of execution with a unique name $\bnfpn{id}$ and a persisted memory, called \textit{context} $\bnfpn{ctx}$.
At a message reception, it executes its function $\bnfpn{fn}$.
This function uses the Javascript syntax augmented with $\bnfpn{stream}$ to indicate an output stream to other fluxions $\bnfpn{dest}$.
The function $\bnfpn{fn}$ is indented to be demarcated from the rest.

\section{Compilation example} \label{section:example}

To illustrate the compiler features, we compiled the example from listing \ref{lst:source} into listing \ref{lst:result}.
Source and result are available on github\cite{flx-example}.
The two asynchronous functions from the source are detected, which result in three \textit{fluxions}.
Callbacks are replaced by output \textit{streams}.
The variable \texttt{res} is read by \textit{fluxion} \texttt{reply-1001}, and is forwarded along the message stream.
The variable \texttt{count} is modified only by \textit{fluxion} \texttt{reply-1001}, and is persisted in its \textit{context}.

\includecode{flx,
  caption={Compilation result for listing \ref{lst:source}},
  label={lst:result}
}
{snippets/result.flx}

\vfill\eject